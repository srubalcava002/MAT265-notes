\documentclass{article}

\begin{document}
\title{Lecture Notes 09/23/2020}
\date{09/23/2020}
\maketitle

\section{Implicit Differentiation}
The thing you gotta do when you have $y$ on the same side of the equation as $x$. While you can just isolate $y$, it's easier to do this, believe it or not. Essentially, any $y$ in the equation is differentiated, but you add a $\frac{dy}{dx}$ to it. If it's just a $y$ by itself, then it turns into $\frac{dy}{dx}$, just like regular derivatives.\\\\
For example:\\
\begin{equation}
\frac{d}{dx}[x^2 + y^2 = 0]
\end{equation}
\[\frac{d}{dx}[x^2] + \frac{d}{dx}[y^2] = \frac{d}{dx}[0]\]
\[2x + 2y(\frac{dy}{dx}) = 0\]
\[2y(\frac{dy}{dx}) = -2x\]
\[\frac{dy}{dx} = \frac{-2x}{2y} \to -\frac{x}{y}\]\\
How to do the hard problem during the lecture:\\\\
\begin{equation}
\frac{d}{dx}[4(x^2 + y^2) = (x^2 + y^2 - 2x)^2]
\end{equation}
\[\frac{d}{dx}[4(x^2 + y^2) = \frac{d}{dx}[(x^2 + y^2 - 2x)^2]\]\\
Use the chain rule on the left side to get:
\[4(2x + y(\frac{dy}{dx}) = 2(x^2 + y^2 - 2x)(2x + 2y(\frac{dy}{dx}) - 2)\]\\
Multiply the polynomials on the left side:
\[4(2x + y(\frac{dy}{dx}) = \]
\end{document}