\documentclass{article}

\title{Lecture Notes 10/02/2020}
\date{10/02/2020}

\begin{document}
\maketitle

\section{The Exponential Function}
\textbf{Definition}: 
\[f(x) = a^x\]
Where $x > 1$ and $a \neq 0$\\
Its domain is $(-\infty, \infty)$ and its range is $(0, \infty)$\\\\
\textbf{Proposition}: $f(x) = a^x$\\
When $a > 1$
\[\lim_{x \to -\infty}a^x = 0\]
\[\lim_{x \to \infty}a^x = \infty\]
When $0 < a < 1$
\[\lim_{x \to -\infty}a^x = \infty\]
\[\lim_{x \to \infty}a^x = 0\]

\subsection{Evalutating Limits of Exponential Functions}
\textbf{Evaluate} $\lim_{x \to \infty}\frac{2}{5^x}$\\\\
Take the two out
\[2\frac{1}{5^x}\]
Direct substitution
\[2(\frac{1}{5^\infty})\]
Simplify
\[2(0) = 0\]\\
\textbf{Evaluate} $\lim_{x \to \infty}\frac{2-5x}{2+3(5)^x}$\\\\
Divide by the highest degree of $x$
\[\frac{5^x(\frac{2}{5^x} - 1)}{5^x(\frac{2}{5^x} + 3)}\]
Simplify and direct substitution
\[\frac{1(\frac{2}{5^\infty} - 1)}{1(\frac{2}{5^\infty} + 3)} = \frac{(0 - 1)}{(0 + 3)} = -\frac{1}{3}\]
\textbf{Evaluate} $\lim_{x \to -\infty} e^{7x}sin(x)$\\
Break into two functions
\[h(x) = e^{7x}, g(x) = sin(x)\]
Use the Squeeze Theorem
\[-1 \leq sin(x) \leq 1\]
\[-e^{7x} \leq e^{7x}sin(x) \leq e^{7x}\]
\textit{Note}: $h(x) = e^{7x}$ can bee seen as the amplitude of $g(x) = sin(x)$\\
Now take the limits
\[\lim_{x \to -\infty}-e^{7x} = 0\]
\[\lim_{x \to \infty}e^{7x} = 0\]
We get
\[\lim_{x \to -\infty}e^{7x}sin(x) = 0\]
\subsubsection{Cool identities}
\[a^{x+y} = a^x + a^y\]
\[a^{x-y} = \frac{a^x}{a^y}\]
\[(a^x)^y = a^{xy}\]
\[(ab)^x = a^x b^x\]
\end{document}