\documentclass{article}
\begin{document}

\title{MAT 256 EXAM 1 NOTES}
\maketitle

\section{Domain of a function}

\subsection{Rational Functions}
Basically, the denominator can never equal zero.\\
\\
For example:\\
$\frac{1}{x^2+5x+6}$ has a domain of $x \neq -2$ and $x \neq -3$ because the bottom polynomial is equal to 0 when x is either -2 or -3. Solve the polynomial yourself and see.

\subsection{Radical Functions}
For these, never let what's inside the square root equal anything less than zero, because math gets weird when you have a negative in the square root.\\
\\
For example:\\
$\sqrt{x-3}$ has a domain of $[3, \infty)$ because anything smaller than 3 makes the square root negative.

\subsection{Combined}
Take both principles and apply them to the function:\\
$\frac{\sqrt{15-3x}}{x+2}$\\
\\
The domain is $(-\infty, -2) \cup (-2, 5]$ because \textit{EXACTLY} -2 makes the denominator 0 and anything bigger than 5 makes the square root negative.

\section{Odd and Even Functions}
Just plug in $-x$ for $x$ in your function. If:\\
\\
$f(-x) = f(x)$ (your original function), then the function is even.\\
\\
If you end up with $f(-x) = -f(x)$, then your function is odd.\\
\\
Otherwise, it's neither.

\section{Composite Functions}
\textbf{FILL IN LATER}

\section{Limits}

\subsection{Numerically}
$\lim_{x \to 3} \frac{1}{3-x}$\\
Just plug in numbers that are slightly below 3 (2.9999) and numbers that are above 3 (3.00001). You can put it in your calculator and look at the table.\textit{ Easy peasy}

\subsection{Graphically}
Just look at it lmao. $\lim_{x \to 0^-}$ is the limit from the left and $\lim_{x \to 0^+}$ is the limit from the right.

\subsection{Limits Involving Infinity}
If it's a rational function, divide the denominator by the highest degree of $x$, something something square roots, then simplify.\\
It's the wizard thing that blew your mind.

\section{Squeeze Theorem}
Definition: $\lim_{x \to a} g(x) = L$ if $f(x) <= g(x) <= h(x)$  and both $\lim_{x \to a} f(x)$ and $\lim_{x \to a} h(x) = L$\\
You need two helper functions ($f(x)$ and $h(x)$).\\
\\
How to do it:

\section{Continuity}
There are three conditions a function must meet in order to be continuous at a point, $p$:
\begin{itemize}
\item it must be defined at $p$
\item the limit as x approaches $p$ must exist
\item the limit must equal the value of the function, $f(x)$, at the point ($f(p)$)
\end{itemize}

\section{Derivatives}

\subsection{Definition of a Derivative}
$f'(x) = \frac{f(x+h) - f(x)}{h}$\\
\textbf{PLUG IN $(h + x)$ ANYWHERE YOU SEE AN $x$ IN THE ORIGINAL FUNCTION}

\subsection{Average Rate of Change}
Basically the slope of whatever your function is, i guess.
$\frac{y_{2} - y_{1}}{x_{2}-x_{1}}$

\subsection{Tangent Lines}
To find the tangent line at a given point:
\begin{enumerate}
\item get the derivative of the original function
\item if you're given a point, or at least an x value, plug it into the derivative (\textbf{THIS GETS YOU THE SLOPE, ($m$)})
\item if you're not given a y value, plug the x value into the original equation
\item use point slope formula ($y - y_{1} = m(x - x_{1})$) to get the equation of the tangent line
\item simplify
\end{enumerate}

\subsection{Important Derivative Formulas}
Product rule:\\
$\frac{d}{dx} f(x) * g(x) = f'(x)g(x) + f(x)g'(x)$\\
\\
Quotient rule:\\
$\frac{d}{dx} \frac{f(x)}{g(x)} = \frac{g(x)f'(x) - f(x)g'(x)}{g(x)^2}$\\
\\
Trig stuff:\\
$\frac{d}{dx} sin(x) = cos(x)$\\
$\frac{d}{dx} cos(x) = -sin(x)$\\
$\frac{d}{dx} -sin(x) = -cos(x)$\\
$\frac{d}{dx} -cos(x) = sin(x)$\\
It's like a loop
\end{document}
