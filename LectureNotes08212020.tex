\documentclass{article}

\begin{document}
\title{Lecture Notes}
\date{08/21/2020}
\maketitle

\section{Functions and Limits}

\subsection{Definition of Limit}
Limit - $\lim_{x \to a}f(x)=L$ "The limit of f of x as x approaches a"\\
We are concerned about the points that are around a, not a itself. In other words: a can be undefined.\\
For a limit to exist, both left and right limits MUST converge on the SAME point.
Mathematically: $lim_{x \to a^-}f(x) = \lim_{x \to a^+}f(x)$ where $a^-$ are numbers less than a and $a^+$ are numbers slightly larger than a.\\
Limits must also not approach infinity to exist.

\subsubsection{Exambles of Solving Limits Numerically}
$\lim_{x \to 2}\frac{x^2+x+2}{x+1}$\\
\\
$\frac{\lim_{x \to 2}x^2+x+2}{\lim_{x \to 2}x+1} \to
\frac{2^2+2+2}{2+1} \to \frac{8}{3}$

\subsection{Evaluating Limits Numerically}
$\lim_{x \to -4}\frac{x^2-16}{x+4}$\\
To solve this numerically, you need to plug in numbers that are above and below -4.\\
You end up getting something like -8.\\
Limits can also be evaluated graphically.
\end{document}
